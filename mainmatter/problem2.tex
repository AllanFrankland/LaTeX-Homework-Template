\begin{homeworkProblem}
  Let \(\Sigma = \{0, 1\}\). Construct a DFA \(A\) that recognizes the
  language that consists of all binary numbers that can be divided by 5.
  \\

  Let the state \(q_k\) indicate the remainder of \(k\) divided by 5. For
  example, the remainder of 2 would correlate to state \(q_2\) because \(7
  \mod 5 = 2\).

  \begin{figure}[h]
      \centering
      \begin{tikzpicture}[shorten >=1pt,node distance=2cm,on grid,auto]
          \node[state, accepting, initial] (q_0)   {$q_0$};
          \node[state] (q_1) [right=of q_0] {$q_1$};
          \node[state] (q_2) [right=of q_1] {$q_2$};
          \node[state] (q_3) [right=of q_2] {$q_3$};
          \node[state] (q_4) [right=of q_3] {$q_4$};
          \path[->]
              (q_0)
                  edge [loop above] node {0} (q_0)
                  edge node {1} (q_1)
              (q_1)
                  edge node {0} (q_2)
                  edge [bend right=-30] node {1} (q_3)
              (q_2)
                  edge [bend left] node {1} (q_0)
                  edge [bend right=-30] node {0} (q_4)
              (q_3)
                  edge node {1} (q_2)
                  edge [bend left] node {0} (q_1)
              (q_4)
                  edge node {0} (q_3)
                  edge [loop below] node {1} (q_4);
      \end{tikzpicture}
      \caption{DFA, \(A\), this is really beautiful, ya know?}
      \label{fig:multiple5}
  \end{figure}

  \textbf{Justification}
  \\

  Take a given binary number, \(x\). Since there are only two inputs to our
  state machine, \(x\) can either become \(x0\) or \(x1\). When a 0 comes
  into the state machine, it is the same as taking the binary number and
  multiplying it by two. When a 1 comes into the machine, it is the same as
  multipying by two and adding one.
  \\

  Using this knowledge, we can construct a transition table that tell us
  where to go:

  \begin{table}[ht]
      \centering
      \begin{tabular}{c || c | c | c | c | c}
          & \(x \mod 5 = 0\)
          & \(x \mod 5 = 1\)
          & \(x \mod 5 = 2\)
          & \(x \mod 5 = 3\)
          & \(x \mod 5 = 4\)
          \\
          \hline
          \(x0\) & 0 & 2 & 4 & 1 & 3 \\
          \(x1\) & 1 & 3 & 0 & 2 & 4 \\
      \end{tabular}
  \end{table}

  Therefore on state \(q_0\) or (\(x \mod 5 = 0\)), a transition line should
  go to state \(q_0\) for the input 0 and a line should go to state \(q_1\)
  for input 1. Continuing this gives us the Figure~\ref{fig:multiple5}.
\end{homeworkProblem}